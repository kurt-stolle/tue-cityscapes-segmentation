\subsection{The dataset}
\label{sec:dataset}

\begin{table}
    \centering
    \caption{Target labels}
    \label{tab:labels}
    \begin{tabular}{ll}
        \hline
        Category    & Labels \\
        \hline
        Flat        & road, sidewalk, parking, rail track \\
        Human       & person, rider \\
        Vehicle     & car, truck, bus, on rails, motorcycle, bicycle, trailer \\
        Construction& building, wall, fence, guard rail, bridge, tunnel \\
        Object      & pole, pole group, traffic sign, traffic light \\
        Nature      & vegetation, terrain \\
        Sky	        & sky \\
        Void        & ground, dynamic, static \\
        \hline
    \end{tabular}
\end{table}

As with any machine learning project, first the available data has to be parsed into a practical format.

The dataset consists of a collection of PNG-encoded images (the input) and a corresponding segmentation mask (the ground truth). The segmentation mask is formatted as a JSON-document containing a list of objects that pair a label with a polygon that describes the shape of this object in the image.

When an item in the dataset is requested, the PNG-file is read and converted to an array of matrices, where each channel (red, green and blue) is an element of the array. 
The polygons file corresponding to this item is then loaded and the polygons are converted to rasters and combined into a single image, with each pixel of this image corresponding to one label (see \Cref{tab:labels}) or nothing.

\subsection{Setting a baseline}
\label{subsec:baseline}

The baseline implementation sets a reference point to improve upon and score our method against.

The architecture U-Net is suitable for semantic segmentation of biological cells under a microscope~\cite{RonnebergerFB15}. 
In this paper, the same implementation is used as a baseline for the segmentation of the cityscapes.

Our implementation of the UNet is based on~\cite{GH-Pytorch-UNet2018} with some modifications to work with the Cityscapes Dataset.

\subsection{Data augmentation}
\label{subsec:data-augmentation}

A straightforward way to improve the accuracy of the model is to increase the size of the training set by applying a set of transforms. The following transforms were used:
\begin{itemize}
    \item Zoom \& rotate
    \item Mirror over the vertical axis
\end{itemize}

\subsection{Network architecture}
\label{subsec:network-architecture}

\subsection{Tweaking hypterparameters}


